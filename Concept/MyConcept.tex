\documentclass[12pt, a4paper]{article}
\pagenumbering{arabic}
\usepackage[utf8]{inputenc}
\usepackage[T1]{fontenc}
\usepackage{geometry}
\usepackage{graphicx}
\graphicspath{{images/}}
\geometry{a4paper}
\usepackage{helvet}
\newcommand\tab[1][1cm]{\hspace*{#1}}
\renewcommand{\familydefault}{\sfdefault}
\setlength{\topmargin}{-2cm}
\setlength{\oddsidemargin}{0cm}
\setlength{\textheight}{24cm}
\setlength{\textwidth}{16cm}
\usepackage{graphicx}
\usepackage{listings}
\usepackage{sectsty}
%Use Helvetica as the sansserif font
\usepackage{helvet}
%Use sffamily for all titles
\allsectionsfont{\sffamily}
\title{ENERGY FOR COOKING IN DEVELOPING COUNTRIES : FOCUS ON UGANDA}
\author{Prepared by Stephen Gift Mukoya Araka}
\date{$24^{th} Feb$ $2018$}
\begin{document}
\maketitle
\clearpage
\section{Introduction}
\subsection{Research background}
In developing countries, especially in rural areas, 2.5 billion people rely on biomass, such as fuelwood, charcoal, agricultural waste and animal dung, to meet their energy needs for cooking. In many countries, these resources account for over 90\% of household energy consumption.

In the absence of new policies, the number of people relying on biomass will increase to over 2.6 billion by 2015 and to 2.7 billion by 2030 because of population growth. That is, one-third of the world’s population will still be relying on these fuels. There is evidence that, in areas where local prices have adjusted to recent high international energy prices, the shift to cleaner, more efficient use of energy for cooking has actually slowed and even reversed.
 
\subsection{Research Findings \& Problem statement}
In Uganda, 90\% of households use firewood, charcoal for cooking.
Almost every household in the country either uses charcoal or firewood for cooking, the 2016/2017 Uganda National Household Survey has noted, revealing that the dependency on firewood as a source of energy is straining the environment.

According to Vincent Fred Ssennono, a lead researcher at the Uganda Bureau of Statistics (UBOS), a large portion of households using wood fuel for cooking, is in northern, Karamoja, West Nile and Kigezi sub regions.
“Over 90\% of the households use wood fuel for cooking,” Ssennono said, urging the National Environment Management Authority (NEMA) to devise interventions that might save the depletion of forests in the country.

\subsection{Aim and objectives}
\subsubsection{Aim or general objective}
To assess or analyse the quality of energy used for cooking and to replace pollutant wood fuel with clean burning energy stoves.
\subsubsection{Specific objectives}
To compare  the current energy available for use by the population of Uganda to the suggested source (JikoKoa).
To determine the possibiity of families or individuals to purchase cleaner energy sources for cooking through savings \& loans.
To reduce indoor and outdoor pollution and save energy.
Compared to the noemal wood fuels used, JikoKoa lights 50\% faster, saves 60\% charcoal, and produces 63\% less smoke!


\subsection{Research significance}
The purpose of this study is to find out the impact of the wood fuel currently being used by majority of the population in Uganda and to find out the solutions that can be undertaken in order to improve energy usage and for comfortable better living by the population.




\subsection{Research scope}
This study focuses on the entire population of the country but majorly on the rural areas in Uganda (being the popular areas where wood fuel is being used).

\section{Methodology}
This research is going to be carried out through making interviews with people, use of questionnaires, recording down important information on ground and the use of  the  ODK collect application to capture images and audios about the current energy being used for cooking by the people and their views on clean energy (JikoKoa).
\clearpage

\Large{\textbf{References}}\\

\normalsize https://burnstoves.com/jikokoa/\\
The Green Elephant (Smart Energy Solutions) :
http://thegreenelephant.org/2017/09/27/90-of-households-use-firewood-charcoal-for-cooking/
https://www.iea.org/publications/freepublications/publication/cooking.pdf

\end{document}